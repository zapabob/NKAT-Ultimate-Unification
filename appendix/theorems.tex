\section{Mathematical Theorems}

\subsection{Noncommutative Information Conservation Theorem}

\begin{theorem}[Noncommutative Information Conservation]
For any quantum system described by NKAT, the following conservation law holds:
\[
\frac{d}{dt} \int d^3x\, (\hat{I} \cdot \hat{F}^{0i}_{\text{NQG}}) = 0
\]
\end{theorem}

\begin{proof}
The proof follows from the unitary evolution of quantum states and the noncommutative structure of the information field.
\end{proof}

\subsection{Information-Existence Entanglement Theorem}

\begin{theorem}[Information-Existence Entanglement]
For any physical state \(|\psi\rangle\), there exists an entangled state with the information field \(\mathcal{I}\):
\[
|\Psi\rangle = \sum_{i,j} \alpha_{ij} |\psi_i\rangle_{\text{existence}} \otimes |\mathcal{I}_j\rangle_{\text{info}}
\]
\end{theorem}

\begin{proof}
This follows from the fundamental structure of NKAT and the properties of quantum entanglement.
\end{proof}

\subsection{Minimum Length Scale Theorem}

\begin{theorem}[Minimum Length Scale]
The NKAT framework predicts a minimum length scale:
\[
l_{\text{min}} = \sqrt{\theta} \sim 10^{-33}\,\text{cm}
\]
\end{theorem}

\begin{proof}
This is derived from the noncommutative structure of spacetime and the uncertainty principle.
\end{proof} 